% cd /disks/PROJECT/Mickael/COMMUNICATION/SFD2017/;
% pdflatex Report_SFD2017.tex; bibtex Report_SFD2017; pdflatex Report_SFD2017.tex; pdflatex Report_SFD2017.tex;
% evince Report_SFD2017.pdf &

\documentclass[11pt,a4paper,sans]{moderncv}
\moderncvstyle{casual}
\moderncvcolor{blue}

\usepackage[margin=2.5cm]{geometry}
\setlength{\footskip}{40pt}

\usepackage[french,english]{babel}
\selectlanguage{french}

\usepackage[super]{natbib}
\usepackage{indentfirst}

\newcommand\cmd[1]{{\texttt{\color{black}\textbf{#1}}}}

\firstname{}
\familyname{Détection de Nouveaux Variants Génomiques Simultanément Associés au Glucose Sanguin et à la Survenue du Diabète de Type 2.}
\title{\small{\setlength{\parindent}{0cm}\textbf{\underline{Mickaël Canouil}}\textsuperscript{1}, Philippe Froguel\textsuperscript{1,2,\textasteriskcentered}, Ghislain Rocheleau\textsuperscript{1,\textdagger}\\
\normalsize
\textsuperscript{1}Université de Lille , CNRS, Institut Pasteur de Lille, UMR 8199 - EGID, Lille, France\\
\textsuperscript{2}Department of Genomics of Common Disease, Imperial College London, London, United Kingdom\\
\textsuperscript{\textasteriskcentered}Directeur de thèse. \textsuperscript{\textdagger}Co-Directeur de thèse.
}}
\address{EGID - UMR 8199 - Pôle Recherche - 1er étage Aile Ouest}{1, Place de Verdun, 59045 LILLE CEDEX, France}
\phone{+33 (0) 374 00 81 29}
\email{mickael.canouil@cnrs.fr}
\homepage{mickael.canouil.fr}{mickael.canouil.fr} % The first argument is the url for the clickable link, the second argument is the url displayed in the template - this allows special characters to be displayed such as the tilde in this example

\begin{document}
\makecvtitle % Print the CV title

\section{Avancées Principales}
\par{\setlength{\parindent}{0.5cm}
L’objectif principal de ce projet consiste en la découverte de nouveaux loci simultanément associés à la trajectoire temporelle du glucose sanguin et à l’incidence du diabète de type 2 (DT2),
via l'implémentation d'une approche par modèle joint combinant des mesures longitudinales d’un même trait (glycémie à jeun)
et des temps de survenue d’une maladie (temps au diagnostic du DT2) dans un modèle statistique unique\cite{ibrahim_basic_2010, tsiatis_joint_2004}.\newline
Combiner des données longitudinales (p.ex., à l’aide d’un modèle linéaire mixte) et des données de survenue d’un événement (p.ex., à l’aide d’un modèle de risque proportionnel de Cox)
représente une stratégie plus puissante puisqu’elle prend en compte la dépendance entre ces deux types de données.
Par exemple, cette stratégie est plus efficace que des approches plus naïves telles que l’introduction des mesures longitudinales comme covariables variant dans le temps directement dans un modèle de Cox,
ou encore l’utilisation d’une approche en deux étapes\cite{tsiatis_modeling_1995} où, dans une première étape, un modèle linéaire mixte est ajusté aux mesures longitudinales puis, dans une deuxième étape,
la fonction résultante introduite comme covariable variant dans le temps dans un modèle de Cox.}
\vspace{0.5em}
\par{\setlength{\parindent}{0.5cm}
Le déroulement de ce projet se décomposait principalement en deux tâches, consistant en l'implémentation d'un modèle joint et à l'application de ce modèle à un jeu de données réelles.
% \vspace{0.25em}

\begin{description}
    \setlength{\itemsep}{0.5em}
    \item[\textbf{Tâche 1:}] L'extension R: \cmd{JM} \cite{rizopoulos_jm_2010} a été implémentée via la parallélisation des modèles statistiques et des principales fonctions,
        afin de réduire le temps de calcul et permettre l'application à l'échelle pan-génomique. Les approches plus naïves, modèle de Cox avec covariables dépendantes du temps et approche en deux étapes, ont également été implémentées.
        Les performances statistiques et computationnelles de ces différentes approches ont été étudiées via simulation.

    \item[\textbf{Tâche 2:}] Les \mbox{4 352} individus (167 DT2 incidents) de la cohorte D.E.S.I.R.\cite{balkau_epidemiologic_1996} et \mbox{101 165} SNPs (Illumina Metabochip\cite{voight_metabochip_2012}) ont été analysés par un modèle joint et par des approches transversales classiques,
        telles que des régressions linéaires et logistiques pour tester l'effet des variants génétiques sur le niveau de glucose sanguin (à l'inclusion) et sur le risque de DT2 (DT2 à l'inclusion et incidents).
        La puissance statistique, l'erreur de type 1 et le temps de calcul ont été mesurés, puis comparés entre notre approche et les approches transversales.
\end{description}
}
\vspace{0.5em}
\par{\setlength{\parindent}{0.5cm}
La réalisation des \textbf{Tâche~1} et \textbf{Tâche~2} a permis l'obtention de résultats théoriques et appliqués démontrant les avantages des approches par modèle joint, 
par rapport aux approches traditionnelles utilisées dans les études d'associations pan-génomiques.\newline
Ainsi, nous avons pu montrer des résultats cohérents avec la littérature, notamment pour les loci au niveau de G6PC2/ABCB11, GCK/YKT6 et GCKR, avec un nombre d'individus plus faible par l'utilisation de mesures répétées. 
La modélisation jointe a également permis d'identifier de façon plus fines les associations, par exemple, au niveau du locus MTNR1B (rs10830963\_G), où nos résultats indiquent que l'allèle à risque pourrait diminuer le risque de DT2, tout en augmentant la glycémie à jeun au cours du temps.\newline
Les principaux résultats ont fait l'objet de présentations lors de congrès internationaux et feront l'objet d'une publication, dans les prochains mois, dans une revue scientifique à comité de lecture.
}

\clearpage
\section{Communications}
\subsection{Présentations Orales}
\begin{itemize}
    \setlength{\itemsep}{0.5em}
    \item \textbf{Longitudinal Genetic Modelling: Revisiting Associations of SNPs Associated with Blood Fasting Glucose in Normoglycemic Individuals}
        \newline \underline{Mickaël Canouil}, Ghislain Rocheleau, Loïc Yengo and Philippe Froguel
        \newline \textit{Statistical Methods for Post Genomic Data - SMPGD, Lille, France (2016)}
\end{itemize}

\subsection{Présentations Poster}
\begin{itemize}
    \setlength{\itemsep}{0.5em}
    \item \textbf{Application of Joint Models in Genetic Association Studies}
        \newline Ghislain Rocheleau, \underline{Mickaël Canouil}, Loïc Yengo and Philippe Froguel
        \newline \textit{International Genetic Epidemiology Society - IGES, Baltimore, États-Unis (2015)}
    \item \textbf{Single Nucleotide Polymorphisms Associated With Fasting Blood Glucose Trajectory And Type 2 Diabetes Incidence: A Joint Modelling Approach}
        \newline \underline{Mickaël Canouil}, Philippe Froguel and Ghislain Rocheleau
        \newline \textit{International Genetic Epidemiology Society - IGES, Toronto, Canada (2016)}
    \item \textbf{Single Nucleotide Polymorphisms Associated With Fasting Blood Glucose Trajectory And Type 2 Diabetes Incidence: A Joint Modelling Approach}
        \newline \underline{Mickaël Canouil}, Philippe Froguel and Ghislain Rocheleau
        \newline \textit{4th Symposium European Genomic Institute for Diabetes (E.g.i.d), Lille, France (2016)}
    \item \textbf{Variants Génétiques Associés à la Trajectoire de la Glycémie à Jeun et à l’Incidence du Diabète de Type 2: Une Approche par Modèle Joint} (CA-075)
        \newline \underline{Mickaël Canouil}, Philippe Froguel and Ghislain Rocheleau
        \newline \textit{Congrès Annuel de la Société Francophone du Diabète (SFD), Lille, France (2017)}
\end{itemize}

\bibliographystyle{apalike}
\bibliography{Mickael_Thesis.bib}

\end{document}
